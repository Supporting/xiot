\documentclass[t]{beamer}
%\usepackage{pgf}
%\usepackage{beamerthemeHannover}
%\usecolortheme{beetle}

\usetheme{PaloAlto}
\usecolortheme{fly}


\usepackage{color}
\definecolor{actorblue}{rgb}{0.0, 0.016, 0.698}
\definecolor{commentColor}{rgb}{0.0, 0.6, 0.0}
\definecolor{listinggray}{gray}{0.95}

\usepackage{listings}
\lstloadlanguages{xml,VRML,c++}

%\lstset{	basicstyle=\ttfamily\small, 
%\lstset{basicstyle=\ttfamily\tiny, keywordstyle=\bfseries}
%\lstset{rulecolor=\color{actorblue}}
%\lstset{linewidth=\textwidth}
%\lstset{commentstyle=\textit, stringstyle=\upshape,showspaces=false,showstringspaces=false} %\lstset{numbers=left,numberstyle=\tiny}
%\lstset{frame=trbl,frameround=ffff}
%\lstset{captionpos=b}
%\lstset{breaklines=true}
%\lstset{aboveskip=\bigskipamount}


\lstset{	basicstyle=\ttfamily\tiny, 
%					backgroundcolor=\color{listinggray},
					identifierstyle=\ttfamily,
					rulecolor=\color{actorblue},
					linewidth=\textwidth,
					commentstyle=\color{commentColor}, 
				  stringstyle=\upshape,
				  showspaces=false,
				  showstringspaces=false,
				  tabsize = 2,
					keywordstyle=\bfseries\ttfamily\color[rgb]{0,0,1},				  
					prebreak = \raisebox{0ex}[0ex][0ex]{\ensuremath{\hookleftarrow}},
					captionpos=b,
					captionstyle=\tiny
			} 

\lstset{frame=trbl,frameround=ffff}


\pgfdeclareimage[height=1cm]{logo1}{F:/Actor3D/images/logos/logo_simple2}


\title{X3D Training Session}
\subtitle{@EDF R\&D}
\author{Kristian Sons}
\logo{\includegraphics[height=1.2cm]{F:/Actor3D/images/logos/logo_simple2}}
%\institution{EDF}
\date{10+11.03.2009}
%\titlegraphic{}

\begin{document}

\begin{frame}[t]
\titlepage
\end{frame}

%\begin{frame}
%\frametitle{Outline}
%\tableofcontents
%\end{frame}
%\AtBeginSection[]
%{
%   \begin{frame}
%       \frametitle{Outline}
%       \tableofcontents[currentsection]
%   \end{frame}
%}

%\part{Introduction}
%\setbeamertemplate{background canvas}[vertical shading][top=blue!60!black, bottom=red!60!black]



\begin{frame}[t]
\frametitle{About me}
\framesubtitle{Kristian Sons}
\begin{itemize}
	\item Media computer scientist
	\item Focus on computer graphics
	\item Scientist at EADS Corporate Research Center 
	\item Experience in
	\begin{itemize}
	\item C++, Java, Python
	\item Eclipse RCP, EMF, Qt
	\item OpenGL, Realtime Raytracing
	\item Shader, VTK, OpenInventor
	\end{itemize}
	\item Founded Actor3D Oct. 2007
	\begin{itemize}
	\item Services and projects: \\EDF R\&D, EADS IW, Teraport
	\item Products: Actor3D Editor
	\end{itemize}
\end{itemize}


\end{frame}

   \begin{frame}
       \frametitle{Outline}
       \tableofcontents
   \end{frame}


\section{X3D}

\begin{frame}[t]
\frametitle{What is X3D?}
\framesubtitle{General}
\begin{itemize}
  \item Standardized scene graph description
  \begin{itemize}
    \item Support of 2D and 3D geometry
    \item Hierarchical grouping
    \item Lights, Material, (Multi-)Texturing, Shader, Animation,
    Interaction\ldots
  \end{itemize}
  \item X3D 3.2 specification defines 215 scenegraph nodes.
  \item Components
  \begin{itemize}
    \item Humanoid animation
    \item NURBS
  \end{itemize}
  \item Profiles
  \begin{itemize}
    \item Interchange
    \item Interactive
  \end{itemize}
  \item Successor of VRML
\end{itemize}
\end{frame}

\subsection{Features}

\frame<1-2>[label=features]
{
\frametitle{What is X3D?}
\framesubtitle{Special features}
\begin{itemize}
  \item<alert@1> Open ISO standard
  \item<alert@2> Encodings
  \onslide*<2> { 
  \begin{itemize}
    \item XML
    \item Classic VRML
    \item Binary
  \end{itemize}}
  \item<alert@3> References
 \onslide*<3> { 
  \begin{itemize}
  \item Reuse of sub-scenegraph
    \item Exactly the same object
    \item Smaller filesize
    \item Possibly smaller memory consumption
  \end{itemize}
  }
  
    \item<alert@4> Prototypes
    \onslide*<4> {
    \begin{itemize}
  \item Reuse of sub-scenegraph
    \item Connected attributes and nodes
    \item Define new objects 
    \end{itemize}
     }

     \item<alert@5> Events and Routing
    \onslide*<5> {
    \begin{itemize}
  \item Allows interaction in combination with sensor and timer nodes
  \item Fields in scene graph can send and/or receive events depending on type 
    \end{itemize}
    } 

        \item<alert@6> Scripting via Scene access interface (SAI)
    \onslide*<6> {
    \begin{itemize} 
  \item ECMAScript (JavaScript)
  \item Java
    \end{itemize}
    } 
    \end{itemize}
    }
    
    \subsection{Encodings}
    
\begin{frame}[fragile]
\frametitle{Encodings}
\framesubtitle{XML}
\begin{itemize}
  \item Human readable
  \item Easy to process and generate
  \item Large data files
\end{itemize}
\begin{lstlisting}[language={xml}]
<X3D version='3.0' profile='Interchange'>
  <Scene>
    <Transform>
      <NavigationInfo headlight='false' avatarSize='0.25 1.6 0.75' type='"EXAMINE"'/>
      <DirectionalLight/>
      <Transform translation='3.0 0.0 1.0'>
        <Shape>
          <Sphere radius='2.3'/>
          <Appearance>
            <Material diffuseColor='1.0 0.0 0.0'/>
          </Appearance>
        </Shape>
      </Transform>
  </Scene>
</X3D>
  \end{lstlisting}

\end{frame}
  
  \begin{frame}[fragile]
\frametitle{Encodings}
  \framesubtitle{Classic VRML}
\begin{itemize}
  \item Human readable
  \item Hard to process
  \end{itemize}
\begin{lstlisting}[language={VRML}]
#X3D V3.2 utf8
PROFILE Interchange

Transform {
  children [
    NavigationInfo {
      headlight FALSE
      avatarSize [ 0.25 1.6 0.75 ]
      type [ "EXAMINE" ]
    }
    DirectionalLight {
    }
    Transform {
      translation 3.0 0.0 1.0
      children [
    Shape {
	  geometry Sphere { radius 2.3
	  }
	  appearance Appearance {
	    material Material { diffuseColor 1.0 0.0 0.0 }
	  }
	}
    ] } ] }
\end{lstlisting}

\end{frame}
  
  \begin{frame}[t]
\frametitle{Encodings}
  \framesubtitle{Binary}
\begin{itemize}
  \item Binary encoding of XML using Fast Infoset (FI)
  \item X3D-specific field encodings
  \item Lossy or lossless
  \item Not human readable
  \item Small filesizes
  \item Fast processing possible
  \item Project with EDF has shown: \\ Compression rate up to 95\% compared to
  XML encoding 
  \end{itemize}
  \end{frame}
  
  \againframe<3-6>{features}
  
  \subsection{Applications}
  
 \begin{frame}
 \frametitle{What is X3D?}
\framesubtitle{Applications}
\begin{itemize}
  \item Product Presentation
  \item Product Configuration
  \item Manuals / Training
  \item Games / Entertainment
  \item Scientific Visualization 
  \item Visualize and Communicate Simulation Results
  \item \dots
\end{itemize}
 \end{frame}
  
    
 \begin{frame}
 \frametitle{What is X3D?}
\framesubtitle{Media}
\begin{itemize}
  \item Desktop
  \item Web
  \item Mobile
  \item POS / POI / CDROM
  \item Virtual Reality
  \item Embedded in other documents (PDF, Word, PowerPoint)
  \item \dots
\end{itemize}
 \end{frame}
  
  \section{General}
  \subsection{History} 
\begin{frame}
\frametitle{History}
\framesubtitle{Project History}

\begin{itemize}
	\item Project: Binary X3D exporter for VTK
	\onslide*<2> { 
	  \begin{itemize}
		\item Without Java bindings
		\item Reimplementation of existing vtkX3DExporter
		\item One Interface for both encodings, XML and FI
		\item Using VTK data types
		\item Contributed without major changes to VTK CVS tree
		\end{itemize}
	 }
  \item Project: Generalized X3D exporter
	\onslide*<3> { 
  \begin{itemize}
	\item Replace VTK data types with standard C++/STL types
	\item Improved interface
	\item Improved performance
	\end{itemize}
	}
  \item Project: X3D Loader library
	\onslide*<4> { 
  \begin{itemize}
	\item XML and FI encoding
	\item Event based
	\item Demonstrator and Tutorial
	\item Spin-Off: OpenFI-library
	\end{itemize}
	}
  \end{itemize}
\end{frame}
  
\subsection{Concepts} 
\begin{frame}
\frametitle{Concepts}
%\framesubtitle{Workflow}
\begin{itemize}
	\item Hide encoding from user
	\item Use IDs instead of strings
	\item Try to avoid data structures
	\item Keep it simple
\end{itemize}
\end{frame}

\subsection{Libraries}
\begin{frame}
\frametitle{Libraries}
\framesubtitle{Contributions}
\begin{itemize}
	\item Exporter
	\begin{itemize}
		\item zlib
	\end{itemize}
	\item Loader
	\begin{itemize}
		\item zlib, xercesc, openFI
	\end{itemize}
\end{itemize}
\end{frame}


\section{X3D Exporter}

\begin{frame}[fragile]
	\frametitle{X3D Exporter}
	\framesubtitle{Functionality}
	\begin{itemize}
	\item What does the importer for you?
	\begin{itemize}
		\item Write the document using the specified encoding
		\item Document header / footer generation
		\item Formatting (XML)
	\end{itemize}
	\item<2-> You have to take care for semantics:
	\begin{itemize}
		\item Matching start/end Node
		\item Allowed attributes
		\item Allowed data types
	\end{itemize}
	\end{itemize}
\end{frame}

\begin{frame}[fragile]
\frametitle{X3D Exporter}
	\framesubtitle{Code}
\begin{lstlisting}[language={c++}]
writer->StartDocument();
  writer->StartNode(X3D::X3D);
    writer->SetSFString(X3D::profile, "Immersive");
    writer->SetSFString(X3D::version, "3.0");
    writer->StartNode(X3D::Scene);
      writer->StartNode(X3D::Shape);
        writer->StartNode(X3D::Box);
          writer->SetSFVec3f(X3D::size, 2.0f, 0.5f, 0.5f); 
        writer->EndNode();
      writer->EndNode();
    writer->EndNode();
  writer->EndNode();
writer->EndDocument();
\end{lstlisting}
\end{frame}

\begin{frame}[fragile]
\frametitle{X3D Exporter}
	\framesubtitle{Result - XML}
\begin{lstlisting}[language={xml}, title=Result XML encoded: 173 Byte]
<?xml version="1.0" encoding ="UTF-8"?>

<X3D profile="Immersive" version="3.0">
  <Scene>
    <Shape>
      <Box size="2 0.5 0.5"/>
    </Shape>
  </Scene>
</X3D>
\end{lstlisting}
\end{frame}

\begin{frame}[fragile]
\frametitle{X3D Exporter}
	\framesubtitle{Result - FI}
\begin{lstlisting}[title=Result Fast Infoset encoded: 82 Byte]
0000000 010000e0 16001020 3a6e7275 65747865  >.... ...urn:exte<
0000020 6c616e72 636f762d 6c756261 60797261  >rnal-vocabulary`<
0000040 48af406a 6d6d4900 69737265 fa406576  >j@.H.Immersive@.<
0000060 302e3342 200a90f0 a000a056 0008295d  >B3.0... V...])..<
0000100 2e302032 2e302035 f0a0f035 f0a0f0a0  >2 0.5 0.5.......<
0000120 0000ffa0                             >....<
\end{lstlisting}
\end{frame}


\begin{frame}
	\frametitle{Discussion}
	\framesubtitle{What do you think about?}
	\begin{itemize}
		\pause \item Scene Optimizations (s. Chinsel)
		\pause \item Scene reduction
		\pause \item Importer (i.e. OpenJT, fbx...)
		\pause \item Exporter (i.e. U3D)
		\pause \item API
	\end{itemize}
\end{frame}

\section{X3D Loader}

\begin{frame}
	\frametitle{The End}
	\begin{block}{Thank you for your attention!}
	Kristian Sons \\
	Chiemgaustr 62 \\
	81549 M\"{u}nchen / Germany \\ 
	\vspace{1cm}
	
	\begin{tabular}{ll}
	tel  &   +49-89-20327854\\
fax   &  +49-89-20327853\\
mobile & +49-151-50987608\\
mail  &  kristian.sons@actor3d.com\\
web   &  \hyperlink{http://www.actor3d.com}{http://www.actor3d.com}
\end{tabular}
	\end{block}
\end{frame}

\end{document}
